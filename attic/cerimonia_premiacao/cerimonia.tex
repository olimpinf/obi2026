
\documentclass{article}
\usepackage[utf8]{inputenc}
\usepackage{color}
\usepackage[a4paper, total={7in, 9in}]{geometry}
\begin{document}
\setlength{\parindent}{0pt}

\begin{center}
    \textbf{
        UNIVERSIDADE ESTADUAL DE CAMPINAS\\
        CERIMONIAL\\
        \bigskip
         PREMIAÇÃO DA OLIMPÍADA BRASILEIRA DE INFORMÁTICA\\
        INSTITUTO DE COMPUTAÇÃO\\
        DIA – 05 DE DEZEMBRO DE 2025\\
        HORÁRIO: 19H00 ÀS 20H30\\
        IC - Unicamp – CAMPINAS/SP 
    }
\end{center}

\textbf{Mestre de Cerimônia:}\
Boa noite a todos!\
\textbf{Sejam bem vindos à Cerimônia de Premiação da Vigésima Quinta Olimpíada Brasileira de Informática.}\\
\color{blue}Este evento é promovido pela Sociedade Brasileira de Computação. Em sua 27ª edição, a OBI foi organizada pelo Instituto de Computação da Unicamp (IC) sob a Coordenação do Prof. Dr. Ricardo de Oliveira Anido.\\
\color{black}
Convidamos para compor a mesa de honra desta Cerimônia as seguintes autoridades:
\begin{itemize}
%\item \textbf{Prof. Dr. Munir Skaf}\\\\
%Pró-reitor de Pesquisa da Unicamp\\\\
%Professor Titular do Instituto de Química da Unicamp

%\item \textbf{Prof. Dr. Marcelo Duduchi}\\\\
%Diretor da Sociedade Brasileira de Computação (SBC)

\item \textbf{Prof. Dr. Rodolfo Jardim de Azevedo}\\
Ex-Diretor do Instituto de Computação da Unicamp e co-coordenador da Olimpíada Brasileira de Informática 2025.
\item \textbf{Prof. Dr. Ricardo de Oliveira Anido}\\
Professor do Instituto de Computação da Unicamp e \\
Coordenador da Olimpíada Brasileira de Informática 2025
\end{itemize}


Com a palavra o Prof. Dr. Ricardo de Oliveira Anido, professor do Instituto de Computação da Unicamp e Coordenador da OBI 2025.\\
\color{red}\textbf{(Breve discurso de abertura)}\color{black}\\

Faremos neste momento a chamada para a premiação dos alunos, que deverão apresentar-se à frente do palco, para receberem suas medalhas e certificados a serem entregues pelos componentes da mesa.\\
\color{red}\textbf{(Cada slide representa um conjunto de alunos que será chamado simultaneamente ao palco)}\color{black}\\

\color{blue}
\textbf{Iniciaremos a premiação com a Modalidade Iniciação Nível 1, para alunos até o sétimo ano do Ensino Fundamental}\\\\\\
\textbf{\color{blue}Ganhadores de Medalhas de Ouro, \color{black}Modalidade Iniciação Nível 1}
\color{black}

                        \{itemize}\item\n\textbf{14º lugar} - \textbf{Matheus Oliver Berger} – Colégio Móbile – São Paulo/SP


\textbf{14º lugar} - \textbf{Joao Emanuel Cardoso Sales} – ARI DE SÁ CAVALCANTE SEDE MAJOR FACUNDO – Fortaleza/CE



\item\n\textbf{14º lugar} - \textbf{Guilherme Moura de Oliveira} – Escola de Referência em Ensino Fundamental e Médio Aplicação do Recife – Recife/PE


\textbf{14º lugar} - \textbf{Giovani Vieira de Oliveira} – Colégio Ábaco – São Bernardo do Campo/SP


\textbf{14º lugar} - \textbf{Daniel Kenzo Hayashi} – Colégio Etapa - Vila Mariana – São Paulo/SP



\item\n\textbf{14º lugar} - \textbf{Daniel Alves Mendes} – Colégio Militar do Rio de Janeiro – Rio de Janeiro/RJ


\textbf{14º lugar} - \textbf{Daniel Alves Breijão} – Escola Educação Criativa – Ipatinga/MG


\textbf{14º lugar} - \textbf{Caio do Nascimento Rodrigues} – COLÉGIO AGOSTINIANO SÃO JOSÉ – São Paulo/SP



\item\n\textbf{8º lugar} - \textbf{Raul Batista Pereira Wakahara} – Colégio Santa Cruz – São Paulo/SP


\textbf{8º lugar} - \textbf{Lucca Fontes Aragão} – COLÉGIO MILITAR DE FORTALEZA – Fortaleza/CE



\item\n\textbf{8º lugar} - \textbf{Letícia Nicoletti Capriles} – Colégio Militar do Rio de Janeiro – Rio de Janeiro/RJ


\textbf{8º lugar} - \textbf{Francisco Josue Oliveira Fernandes} – Colégio Farias Brito Sobralense – Sobral/CE


\textbf{8º lugar} - \textbf{Davi Luiz Alves Pilé Silva} – Escola de Referência em Ensino Fundamental e Médio Aplicação do Recife – Recife/PE



\item\n\textbf{1º lugar} - \textbf{Saulo Ribeiro Amaral} – Colégio Pedro II - Campus Humaitá II – Rio de Janeiro/RJ


\textbf{1º lugar} - \textbf{Matheus de Lima Barros} – Colégio Militar do Rio de Janeiro – Rio de Janeiro/RJ



\item\n\textbf{1º lugar} - \textbf{Luiz Pedro Araujo Souza} – CENTRO DE EDUCAÇÃO INTEGRADA S.A - RF – Natal/RN


\textbf{1º lugar} - \textbf{Joubert Yuri da Silva Delmondes} – Escola de Referência em Ensino Fundamental e Médio Aplicação do Recife – Recife/PE


\textbf{1º lugar} - \textbf{Felipe Somavila Bley} – Escola de Educação Básica da URI - Santo Ângelo – Santo Ângelo/RS



\end{itemize}\n
\color{blue}
\textbf{Passamos agora à Modalidade Iniciação Nível 2, para alunos até o nono ano do Ensino Fundamental}\\\\\\
\textbf{\color{blue}Ganhadores de Medalhas de Ouro, \color{black}Modalidade Iniciação Nível 2}
\color{black}

                        \{itemize}\item\n\textbf{12º lugar} - \textbf{Raphael Kara José Fernandes} – Colégio Móbile – São Paulo/SP


\textbf{12º lugar} - \textbf{Rafael Costa de Morais} – COLÉGIO GGE – Recife/PE


\textbf{12º lugar} - \textbf{Leonardo Corrêa Netto de Oliveira} – Colégio Franciscano Pio XII – São Paulo/SP



\item\n\textbf{12º lugar} - \textbf{Gabriel Satoshi Albuquerque Trentini} – COLÉGIO MILITAR DE FORTALEZA – Fortaleza/CE


\textbf{12º lugar} - \textbf{Bruno Pordeus Barroso Jales} – COLÉGIO MILITAR DE FORTALEZA – Fortaleza/CE


\textbf{12º lugar} - \textbf{Alexandre Akihiro Kuya} – Colégio Etapa - Vila Mariana – São Paulo/SP



\item\n\textbf{4º lugar} - \textbf{Mikael Ribeiro do Nascimento} – ARI DE SÁ CAVALCANTE SEDE MAJOR FACUNDO – Fortaleza/CE


\textbf{4º lugar} - \textbf{Nelson Morgado Merlo} – Colégio Agostiniano Mendel – São Paulo/SP


\textbf{4º lugar} - \textbf{Max das Colinas Barbosa Martins} – Escola Educação Criativa – Ipatinga/MG



\item\n\textbf{4º lugar} - \textbf{Matheus Matias dos Santos} – ESCOLA ESTADUAL CESÁRIO COIMBRA – Muzambinho/MG


\textbf{4º lugar} - \textbf{Guilherme Martins de Magalhães} – ARI DE SÁ CAVALCANTE SEDE MAJOR FACUNDO – Fortaleza/CE



\item\n\textbf{4º lugar} - \textbf{André Henrique Almeida Gianetti de Souza} – Colégio Marista Arquidiocesano – São Paulo/SP


\textbf{4º lugar} - \textbf{Alice Alencar Spuri} – COLÉGIO ODILON BRAVEZA – Fortaleza/CE



\item\n\textbf{2º lugar} - \textbf{Lorenzo do Prado Oliveira} – Colégio Ábaco – São Bernardo do Campo/SP


\textbf{2º lugar} - \textbf{Danilo Pacheco Couto dos Santos} – Da Vinci Academy – Limeira/SP



\end{itemize}\n
\color{blue}
\textbf{Passamos agora à premiação da Modalidade Programação Nível Júnior, para alunos até o nono ano do Ensino Fundamental}\\\\\\
\textbf{\color{blue}Ganhadores de Medalhas de Ouro, \color{black}Modalidade Programação Nível Júnior}
\color{black}

                        \{itemize}\item\n\textbf{40º lugar} - \textbf{Amanda Sarmento da Silva} – E.M.E.F Dr Julio Casado – Sapucaia do Sul/RS


\textbf{37º lugar} - \textbf{Manuela Farias dos Santos} – COLÉGIO ODILON BRAVEZA – Fortaleza/CE


\textbf{22º lugar} - \textbf{Miguel Asafe de Oliveira Estacio} – COESI - Colégio de Orientação e Estudos Integrados – Aracaju/SE



\item\n\textbf{22º lugar} - \textbf{Caetano Allain} – St Paul's School – São Paulo/SP


\textbf{21º lugar} - \textbf{Júlia Aleixo Melo} – COLÉGIO MILITAR DO RECIFE – Recife/PE


\textbf{17º lugar} - \textbf{Vinícius Knop Gomes} – Colégio Militar de Juiz de Fora – Juiz de Fora/MG



\item\n\textbf{17º lugar} - \textbf{Rafaela Vieira de Melo Coutinho} – COLÉGIO MOTIVA LTDA – Campina Grande/PB


\textbf{17º lugar} - \textbf{Miguel Villaverde Prieto} – "The British School –  Rio de Janeiro"/Rio de Janeiro

\textbf{17º lugar} - \textbf{Daniel Nasser Kahn} – Colégio Santa Cruz – São Paulo/SP



\item\n\textbf{16º lugar} - \textbf{Matheus Daniel Maeda Perin} – Colégio Etapa - Vila Mascote – São Paulo/SP


\textbf{13º lugar} - \textbf{Irina Zhou Ye} – Colégio Etapa - Vila Mariana – São Paulo/SP


\textbf{13º lugar} - \textbf{Francisco Battistella Prado} – Colégio Etapa - Vila Mariana – São Paulo/SP



\item\n\textbf{13º lugar} - \textbf{Augusto Lepage Comello} – Colégio Etapa - Vila Mariana – São Paulo/SP


\textbf{12º lugar} - \textbf{Ana Laura Kogeyama de Faria} – Instituto Educacional de Ensino Fundamental e Médio – São Carlos/SP


\textbf{10º lugar} - \textbf{Bruno Magalhães Mendes} – Escola Educação Criativa – Ipatinga/MG



\item\n\textbf{3º lugar} - \textbf{Pedro Dortas Rocha} – COESI - Colégio de Orientação e Estudos Integrados – Aracaju/SE


\textbf{3º lugar} - \textbf{Rafael Marcos Hachimoto} – Colégio Militar de Curitiba – Curitiba/PR


\textbf{7º lugar} - \textbf{Kevin Neves Ramos Badaró} – Centro Educacional Leonardo da Vinci – Vitória/ES



\item\n\textbf{2º lugar} - \textbf{Guto Mandarino Slapelis} – COESI - Colégio de Orientação e Estudos Integrados – Aracaju/SE


\textbf{2º lugar} - \textbf{Enzo Ricardo Abe} – Colégio Etapa - Vila Mariana – São Paulo/SP


\textbf{2º lugar} - \textbf{Rafael Castro Maia da Cunha} – Centro de Excelência Master – Aracaju/SE



\item\n\textbf{1º lugar} - \textbf{Lucas da Motta Yamaguchi} – Colégio Etapa - Vila Mariana – São Paulo/SP



\end{itemize}\n
\color{blue}
\textbf{Passamos agora à premiação da Modalidade Programação Nível 1, para alunos até o segundo ano do ensino médio}\\\\\\
\textbf{\color{blue}Ganhadores de Medalhas de Ouro, \color{black}Modalidade Programação Nível 1}
\color{black}

                        \{itemize}\item\n\textbf{74º lugar} - \textbf{Malu Araujo Azevedo} – Instituto Federal de Mato Grosso do Sul - Campus Campo Grande – Campo Grande/MS


\textbf{17º lugar} - \textbf{Samuel Nogueira Santiago} – Colégio Ari de Sá sede Aldeota – Fortaleza/CE



\item\n\textbf{17º lugar} - \textbf{Nathan dos Santos Rossi} – Colégio Técnico Industrial de Santa Maria – Santa Maria/RS


\textbf{14º lugar} - \textbf{Vitor Hugo Antunes Gonçalves} – E.E. Antônio Cardoso da Silva – Monte Azul/MG



\item\n\textbf{14º lugar} - \textbf{João Santos Pereira} – COLÉGIO MILITAR DE BELO HORIZONTE – Belo Horizonte/MG


\textbf{14º lugar} - \textbf{Carlos Eduardo Costa de Carvalho} – COLEGIO AMADEUS LTDA – Aracaju/SE


\textbf{11º lugar} - \textbf{Theo Sampaio Ferreira} – COLEGIO SANTO ANTONIO – Belo Horizonte/MG



\item\n\textbf{10º lugar} - \textbf{João Pedro Andrade Brito Santos} – colégio de ciências pura e aplicada – Aracaju/SE


\textbf{8º lugar} - \textbf{Levi Oliveira dos Santos} – COC São Luis – São Luís/MA


\textbf{8º lugar} - \textbf{Carlos Eduardo Mendonça Leite Santana} – Centro de Excelência Master – Aracaju/SE



\item\n\textbf{6º lugar} - \textbf{Igor Brumana de Lima} – Instituto Federal de Minas Gerais - Campus Itabirito – Itabirito/MG


\textbf{5º lugar} - \textbf{David Sztern Cohen} – Escola ORT – Rio de Janeiro/RJ


\textbf{4º lugar} - \textbf{Juan Pedro Ribeiro Hora} – Centro de Excelência Master – Aracaju/SE



\item\n\textbf{2º lugar} - \textbf{Leonardo Faria Lima} – Colégio Etapa - Vila Mascote – São Paulo/SP


\textbf{2º lugar} - \textbf{Gabriel de Carvalho Silva Martins} – Centro de Excelência Master – Aracaju/SE



\item\n\textbf{1º lugar} - \textbf{Rafael Augusto Nascimento Melo} – Centro de Excelência Master – Aracaju/SE



\end{itemize}\n
\color{blue}
\textbf{Passamos agora à premiação da Modalidade Programação Nível 2, para alunos até o terceiro ano do ensino médio}\\\\\\
\textbf{\color{blue}Ganhadores de Medalhas de Ouro, \color{black}Modalidade Programação Nível 2}
\color{black}

                        \{itemize}\item\n\textbf{37º lugar} - \textbf{Rafael Alves Amiune} – Pensi Colégio e Curso – Rio de Janeiro/RJ


\textbf{35º lugar} - \textbf{Talita Ribeiro de Rezende} – Ecol Colegio e Pre-Vestibular LTDA – Campinas/SP



\item\n\textbf{33º lugar} - \textbf{Mateus Salazar Costa} – Centro educacional Adalberto Valle - Unidade Morada do Sol – Manaus/AM


\textbf{33º lugar} - \textbf{Manuella Autran do Nascimento Magalhaes} – Farias Brito - Colégio de Aplicação – Fortaleza/CE



\item\n\textbf{32º lugar} - \textbf{Daniela Emilia Cerda Sales} – Organização Educacional Farias Brito - Pré vestibular Aldeota – Fortaleza/CE


\textbf{31º lugar} - \textbf{Lucas Haruo Kojima} – Colégio Etapa - Vila Mariana – São Paulo/SP


\textbf{29º lugar} - \textbf{Otávio Rocha Pinheiro} – FARIAS BRITO PRE VESTIBULAR CENTRAL – Fortaleza/CE



\item\n\textbf{21º lugar} - \textbf{Sofia Torres de Paula Cintra} – Colégio Etapa - Vila Mariana – São Paulo/SP


\textbf{21º lugar} - \textbf{Marcio Vitor Santos Freitas} – FARIAS BRITO PRE VESTIBULAR CENTRAL – Fortaleza/CE


\textbf{19º lugar} - \textbf{Gustavo Nogueira Mendes} – Colégio Etapa - Vila Mariana – São Paulo/SP



\item\n\textbf{16º lugar} - \textbf{Gabriela Barbieri Stroeh} – Colégio Etapa Valinhos – Valinhos/SP


\textbf{16º lugar} - \textbf{Andre Souza Alberti} – Ifes - Campus Vitória – Vitória/ES


\textbf{15º lugar} - \textbf{Vinícius Reis Souza de Oliveira} – COLÉGIO MILITAR DE BELO HORIZONTE – Belo Horizonte/MG



\item\n\textbf{12º lugar} - \textbf{Tiago Popovic Giavina-bianchi} – Colégio Santa Cruz – São Paulo/SP


\textbf{12º lugar} - \textbf{Lucas Batista Mesquita} – Colégio Etapa - Vila Mariana – São Paulo/SP


\textbf{12º lugar} - \textbf{Ian Carlos Melo Cunha} – COESI - Colégio de Orientação e Estudos Integrados – Aracaju/SE



\item\n\textbf{11º lugar} - \textbf{Levi Magalhaes Pereira Castello Branco} – Colégio Ari de Sá sede Aldeota – Fortaleza/CE


\textbf{10º lugar} - \textbf{Chuan Xi Chen} – Colégio Etapa - Vila Mariana – São Paulo/SP



\item\n\textbf{7º lugar} - \textbf{Matheus Moreira Cabral} – Colégio Bandeirantes – São Paulo/SP


\textbf{7º lugar} - \textbf{Enzo Ivamoto Hirota} – Colégio Etapa - Vila Mariana – São Paulo/SP



\item\n\textbf{6º lugar} - \textbf{João Paulo Campos de Oliveira} – Colégio Método – Belo Horizonte/MG


\textbf{5º lugar} - \textbf{Julia Galdino Tiosso Lopez} – Colégio Etapa - Vila Mariana – São Paulo/SP



\item\n\textbf{4º lugar} - \textbf{Leonardo de Faria Pottes} – Centro de Excelência Master – Aracaju/SE


\textbf{3º lugar} - \textbf{Joao Pedro Gomes Cordeiro de Castro} – Organização Educacional Farias Brito - Pré vestibular Aldeota – Fortaleza/CE



\item\n\textbf{1º lugar} - \textbf{Pedro Suayama Leston Rey} – Organização Educacional Farias Brito - Pré vestibular Aldeota – Fortaleza/CE


\textbf{1º lugar} - \textbf{Maria Clara Fontes Silva} – COESI - Colégio de Orientação e Estudos Integrados – Aracaju/SE



\end{itemize}\n\bigskip

Estes alunos participaram também da Seletiva para a Olimpíada Internacional de Informática de 2025, que acontecerá em agosto do ano que vem, no Uzbequistão. Os alunos realizaram provas diárias e a última delas terminou há alguns minutos. Os alunos selecionados são, em ordem alfabética:\\\\
\color{blue}\textbf{Selecionados para Olimpiada Internacional de Informática}\color{black}

        \begin{itemize}
        \item qualified1\n\item qualified2\n\item qualified3\n\item qualified4\n\end{itemize}\bigskip

Neste momento convidamos para fazer uso da palavra:\\
\color{red}\textbf{(Tempo para discurso: até 3 minutos cada)}\color{black}\\
\begin{itemize}
%\\item Prof. Dr. Munir Skaf, Pró-reitor de Pesquisa da Unicamp
%\\item Prof. Dr. Marcelo Duduchi, Diretor da Sociedade Brasileira de Computação (SBC)
\item Prof. Dr. Rodolfo Jardim de Azevedo, Ex-Diretor do Instituto de Computação da Unicamp e co-coordenador da OBI2025
%%\\item Profa. Dra. Flavia Pisani, Coordenadora Acadêmica do TFC2022
\item Prof. Dr. Ricardo de Oliveira Anido, Coordenador da OBI2025
\end{itemize}
\bigskip

Finalmente faremos um agradecimento especial aos professores e monitores que se dedicaram aos cursos e aos cuidados dos alunos.\\
Nós os chamaremos agora para entrega dos certificados de monitoria e para que recebam os merecidos aplausos:\\

Os Professores:

        \begin{itemize}
        \item André Amaral de Sousa\n\item Arthur Ferreira do Nascimento\n\item Arthur Lobo Leite Lopes\n\item Caique Paiva\n\item Daniel Yuji Hosomi\n\item Emanuel Juliano Morais Silva\n\item Enzo Dantas\n\item Fernando Graminholi Gonçalves\n\item Filipe de Souza Lalic\n\item Heitor Gonçalves Leite\n\item Juliana Borin\n\item Leonardo Valente Nascimento\n\item Luana Amorim\n\item Mateus Bezrutchka\n\item Naim Shaikhzadeh Santos\n\item Pedro César Mesquita Ferreira\n\item Pedro Henrique Assunção\n\item Rafael Nascimento Soares\n\end{itemize}
Os Monitores:

        \begin{itemize}
        \item Wladimir Arturo Garces Carrillo (Coordenador de Monitores OBI2024)\n\item \n\item Amanda Scherr Caldeira Coelho\n\item Athyrson Machado Ribeiro (OBI2024)\n\item Caio Maia Moreira Santos\n\item Evelyn Roberta de Lima Silva (OBI2024)\n\item Guilherme de Godoi Monteiro (OBI2024)\n\item Gustavo Oliveira de Sousa (OBI2024)\n\item Jarol Vijay Butron Soria (OBI2022,OBI2023,OBI2024)\n\item Juliana Costa Dantas\n\item Maria Clara da Costa Oliveira (OBI2023,OBI2024)\n\item Mariana Fonsechi Mandarino\n\item Paulo Bassani (OBI2022,OBI2023,OBI2024)\n\item Thais Steinmüller Farias (OBI2024)\n\item Victoria Alchangelo dos Santos (OBI2023,OBI2024)\n\item Vinicius Leme Soares\n\item Viviane da Silva Pimentel (OBI2024)\n\item \n\item Ana Luiza Mota Gomes\n\item Ana Margarida Borges\n\item Gabriela Taniguchi\n\item Leonardo Ferreira\n\item Lucas Cabral Senno\n\item Lucas Henrique Bertanha\n\end{itemize}\bigskip

Encerramos esta cerimônia, agradecendo a presença dos componentes da mesa de honra e da audiência.\\
	extbf{Parabenizamos mais uma vez a todos os participantes e premiados.}
\end{document}
