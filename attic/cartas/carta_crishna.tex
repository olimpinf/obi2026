\documentclass[11pt,a4paper]{article}
\usepackage[brazil]{babel}
\usepackage{geometry}
\usepackage{eso-pic}
\usepackage[utf8]{inputenc}
\usepackage{epsfig}

\newcommand\BackgroundPic{
  \put(0,0){
    \parbox[b][\paperheight]{\paperwidth}{
      \vfill
      \centering
      \includegraphics[width=\paperwidth,height=\paperheight,keepaspectratio]{letterhead.pdf}
      \vfill
    }
  }
}

\begin{document}
\AddToShipoutPicture{\BackgroundPic}


\thispagestyle{plain}
\pagenumbering{gobble}

\vspace*{2cm}
\noindent
Prof. Thiago Leite\\
Coordenador\\
SENAI-SP\\

\hfill Campinas, 14 de novembro de 2024

\vspace*{1cm}

\noindent
Caro prof. Thiago,
\vspace*{0.5cm}

\noindent
em nome da Olimpíada Brasileira de Informática -- OBI, venho por meio desta
informar que a profa. Crishna Irion
%
%\vfill
%\begin{center}
%\emph{Crishna Irion}
%\end{center}
%\vfill
%
%\noindent
foi convidada para participar como \emph{professora} da Semana Olímpica
da OBI, que ocorrerá no Instituto de Computação da UNICAMP, com os
melhores alunos classificados na OBI, entre os dias 1 e 7 de dezembro
de 2024.

A profa. Crishna tem vasta experiência em competições de programação,
e já lecionou na Semana Olímpica no passado. Essa experiência é fundamental para o
sucesso da Semana Olímpica. Os professores ficam alojados no mesmo
hotel que os alunos e além das aulas também preparam as provas
(diárias), de forma que a dedicação deles é em tempo integral durante
a Semana Olímpica.

Gostaríamos de solicitar a sua colaboração, se possível, para permitir
que a profa. Crisha possa adiar (ou adiantar) a realização de suas atividades acadêmicas,
de modo que possa participar da Semana Olímpica.

A OBI é uma promoção da SBC -- Sociedade Brasileira de Computação, com
coordenação do Instituto de Computação da Unicamp, e reúne anualmente
mais de 100 mil competidores de todo o país. Para mais informações, por
favor consulte\\
 \texttt{http://olimpiada.ic.unicamp.br}.

\vspace*{3mm}

Atenciosamente,

\vfill

\begin{center}
\includegraphics[width=5cm]{signature_blue.jpg}\\
Prof. Dr. Ricardo Anido\\
Instituto de Computação - UNICAMP\\
Coordenador da OBI2024
\end{center}

\vfill

\end{document}
